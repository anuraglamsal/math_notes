\documentclass{article}

\usepackage[left=3cm, right=3cm]{geometry}
\usepackage{graphicx}

\graphicspath{{./}}
\pagenumbering{gobble}

\begin{document}

\section*{Normalizing Constant}

Let's say you have a function like $e^{-x^2}$. This is not a valid Probability Density Function (PDF) because it doesn't satisfy the following property: 
\[\int_{-\infty}^{\infty}f(x)dx=1\]
because:

\[\int_{-\infty}^{\infty}e^{-x^2}dx=\sqrt{\pi}\]
Now, how do we get a function that has the same general shape as our original function, but it satisfies the above property of PDFs. Well, wouldn't dividing $f(x)$ by $\int_{-\infty}^{\infty}f(x)dx$ do it? That is:

\[g(x)=\frac{f(x)}{\int_{-\infty}^{\infty}f(x)dx}\]
In our case:

\[g(x)=\frac{e^{-x^2}}{\sqrt{\pi}}\]
And, obviously:
\[\int_{-\infty}^{\infty}\frac{e^{-x^2}}{\sqrt{\pi}}dx=\frac{1}{\sqrt{\pi}}\int_{-\infty}^{\infty}e^{-x^2}dx=\frac{1}{\sqrt{\pi}}\cdot\sqrt{\pi}=1\]\\
Thus, scaling $f(x)$ with the reciprocal of $\int_{-\infty}^{\infty}f(x)dx$ does the job. Scaling by a constant doesn't change the general shape of the function, so by normalizing, we get to preserve the shape while fulfilling the aforementioned PDF property. Look at the graphs of the function with and without normalization for our particular example:\\

\includegraphics[scale=0.3]{graph}
\centering

\end{document}
